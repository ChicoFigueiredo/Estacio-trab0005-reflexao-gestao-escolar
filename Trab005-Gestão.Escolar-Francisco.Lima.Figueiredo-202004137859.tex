% Comprensão
\pdfminorversion=5
\pdfcompresslevel=9
\pdfobjcompresslevel=2

\documentclass[
	12pt,				% tamanho da fonte
	openright,			% capítulos começam em pág ímpar (insere página vazia caso preciso)
	%twoside,			% para impressão em recto e verso. Oposto a oneside
	oneside,
	a4paper,			% tamanho do papel.
	chapter=TITLE,		% títulos de capítulos convertidos em letras maiúsculas
	section=TITLE,		% títulos de seções convertidos em letras maiúsculas
	sumario=abnt-6027-2012,
	english,			% idioma adicional para hifenização
	brazil				% o último idioma é o principal do documento
]{abntex2}

\label{___:packages}
% ----------------------------------------------------------
% Pacotes básicos
% ----------------------------------------------------------
\usepackage{import}                            % package import tem o comando import que faz a importação com "novo path"
\usepackage{amsmath}							% Pacote matemático
\usepackage{amssymb}							% Pacote matemático
\usepackage{amsfonts}							% Pacote matemático
\usepackage{lmodern}							% Usa a fonte Latin Modern
\usepackage[T1]{fontenc}						% Selecao de codigos de fonte.
\usepackage[utf8]{inputenc}						% Codificacao do documento (conversão automática dos acentos)
\usepackage{lastpage}							% Usado pela Ficha catalográfica
\usepackage{indentfirst}						% Indenta o primeiro parágrafo de cada seção.
\usepackage[dvipsnames,svgnames,table]{xcolor}			% Controle das cores
\usepackage{graphicx}							% Inclusão de gráficos
\usepackage{microtype} 							% para melhorias de justificação
\usepackage{lipsum}								% para geração de dummy text
\usepackage[brazilian,hyperpageref]{backref}	% Paginas com as citações na bibl
%\usepackage[alf]{abntex2cite}					% Citações padrão ABNT
\usepackage[num]{abntex2cite}					% Citações padrão ABNT numérica
\usepackage{adjustbox}							% Pacote de ajuste de boxes
\usepackage{subcaption}							% Inclusão de Subfiguras e sublegendas
\usepackage{enumitem}							% Personalização de listas
\usepackage{siunitx}							% Grandezas e unidades
\usepackage[section]{placeins}					% Manter as figuras delimitadas na respectiva seção com a opção [section]
\usepackage{multirow}							% Multi colunas nas tabelas
\usepackage{array,tabularx} 					% Pacotes de tabelas
\usepackage{booktabs}							% Pacote de tabela profissonal
\usepackage{rotating}							% Rotacionar figuras e tabelas
\usepackage{xfrac}								% Fazer frações n/d em linha
\usepackage{bm}									% Negrito em modo matemático
\usepackage{xstring}							% Manipulação de strings
\usepackage{pgfplots}							% Pacote de Gráficos
\usepackage{tikz}								% Pacote de Figuras
\usepackage[american, cuteinductors,smartlabels, fulldiode, siunitx, americanvoltages, oldvoltagedirection, smartlabels]{circuitikz}						% Pacote de circuitos elétricos
\usepackage{lipsum}

\usepackage{hyperref}
% informações do PDF
\makeatletter
	\hypersetup{
		%pagebackref=true,
		pdftitle={\@title},
		pdfauthor={\@author},
		pdfsubject={\imprimirpreambulo},
		pdfcreator={LaTeX with abnTeX2},
		pdfkeywords={abnt}{latex}{abntex}{abntex2}{trabalho academico},
		colorlinks=true,       		% false: boxed links; true: colored links
		linkcolor=NavyBlue,          	% color of internal links
		citecolor=NavyBlue,        	% color of links to bibliography
		filecolor=black,      		% color of file links
		urlcolor=NavyBlue,
		bookmarksdepth=4,
		linktoc=all
	}
\makeatother


% Centraliza captions of pictures
\usepackage[justification=centering]{caption}


% glossário, com fix para erros
%\usepackage[acronym]{glossaries}


% To use externalize consider
%https://tex.stackexchange.com/questions/182783/tikzexternalize-not-compatible-with-miktex-2-9-abntex2-package
%Lauro Cesar digged into the problem until he came with a solution for me to test. And it Works!
%
%According to this link:
%
%The package calc changed the commands \setcounter and friends to be fragile. So you have to make them robust. The example below uses etoolbox with \robustify:
%
\usepackage{etoolbox}
\robustify\setcounter
\robustify\addtocounter
\robustify\setlength
\robustify\addtolength

\usepackage[]{tocloft}
%\setlength\cftsectionnumwidth{4em}
%\setlength{\cftchapterindent}{2em}
%\setlength{\cftsectionindent}{5em}
%\setlength{\cftsubsectionindent}{8em}

\usepackage{pstricks-add}
\pgfplotsset{compat=1.15}
\usepackage{mathrsfs}
\usetikzlibrary{arrows}

%<<<<<<<WARNING>>>>>>>
% PGF/Tikz doesn't support the following mathematical functions:
% cosh, acosh, sinh, asinh, tanh, atanh,
% x^r with r not integer

% Plotting will be done using GNUPLOT
% GNUPLOT must be installed and you must allow Latex to call external
% programs by adding the following option to your compiler
% shell-escape    OR    enable-write18 
% Example: pdflatex --shell-escape file.tex 


%% How to silence memoir class warning against the use of caption package?
%% https://tex.stackexchange.com/questions/391993/how-to-silence-memoir-class-warning-against-the-use-of-caption-package
%\usepackage{silence}
%\WarningFilter*{memoir}{You are using the caption package with the memoir class}
%\WarningFilter*{Class memoir Warning}{You are using the caption package with the memoir class}


% -----------------------------------------------------------------
% Você pode adicionar seus pacotes a partir desta linha;
% -----------------------------------------------------------------

%\usepackage[showframe,pass]{geometry}
%\usepackage[11,12]{pagesel}
\usepackage{qrcode}
\usepackage{multirow}

% -----------------------------------------------------------------
% pacote interno
\usepackage{modulos/trabalhos.academicos}



% CONFIGURAÇÕES DE PACOTES
% Configurações do pacote backref
% Usado sem a opção hyperpageref de backref
\renewcommand{\backrefpagesname}{Citado na(s) página(s):~}

% Texto padrão antes do número das páginas
\renewcommand{\backref}{}

% Define os textos da citação
\renewcommand*{\backrefalt}[4]{
	\ifcase #1 %
		Nenhuma citação no texto.%
	\or
		Citado na página #2.%
	\else
		Citado #1 vezes nas páginas #2.%
	\fi
}%


\label{___:programinhas}
%-----------------------------------------
% (1) simple command for print or not
%-----------------------------------------
\usepackage{ifthen}
\newcommand{\ImprimirSimOuNao}[2][Sim]
{
  \ifthenelse{\equal{#1}{Sim}}{#2}{}
}

% alterando o aspecto da cor azul
%\definecolor{blue}{RGB}{41,5,195}

\makeatletter
    \newcommand{\includetikz}[1]{%
    	\tikzsetnextfilename{#1}%
    	\input{#1.tex}%
    }
\makeatother


\newcommand{\MONTH}{%
	\ifcase\the\month
	\or JAN% 1
	\or FEV% 2
	\or MAR% 3
	\or ABR% 4
	\or MAI% 5
	\or JUN% 6
	\or JUL% 7
	\or AUG% 8
	\or SET% 9
	\or OUT% 10
	\or NOV% 11
	\or DEZ% 12
	\fi}
\makeatletter

% Personalização das opções das listas
\setlist[itemize]{leftmargin=\parindent}

% Citação online --- MODIFICAR ---
\newcommand{\citeaa}[1]{\citeauthoronline{#1}~(\citeyear{#1})}

\newcommand{\me}[1]{Elaborado pelo autor, #1.}

% Configuração do pgfplots
\pgfplotsset{compat=newest} %compat=1.14
\pgfplotsset{plot coordinates/math parser=false}
\newlength\figureheight
\newlength\figurewidth

% Libraries do TiKz
\usetikzlibrary{quotes,angles,arrows}
\usetikzlibrary{through,calc,math}
\usetikzlibrary{graphs,backgrounds,fit}
\usetikzlibrary{shapes,positioning,patterns,shadows}
\usetikzlibrary{decorations.pathreplacing}
\usetikzlibrary{shapes.geometric}
\usetikzlibrary{arrows.meta}
\usetikzlibrary{external}

%\tikzexternalize[]
%\tikzexternalenable
%\tikzexternalize
%\tikzexternaldisable
%\tikzset{external/force remake}
%\tikzexternalize[shell escape=-enable-write18]

% Configurações do CircuiTiKz
\ctikzset{bipoles/thickness=1}
%\ctikzset{bipoles/length=1.2cm}
\ctikzset{monopoles/ground/width/.initial=.2}
\ctikzset{bipoles/resistor/height=0.25}
\ctikzset{bipoles/resistor/width=0.6}
\ctikzset{bipoles/capacitor/height=0.5}
\ctikzset{bipoles/capacitor/width=0.15}
\ctikzset{bipoles/generic/height=0.25}
\ctikzset{bipoles/generic/width=0.6}
%\ctikzset{bipoles/capacitor polar/length=0.5}
%\ctikzset{bipoles/diode/height=.375}
%\ctikzset{bipoles/diode/width=.3}
%\ctikzset{tripoles/thyristor/height=.8}
%\ctikzset{tripoles/thyristor/width=1}
\ctikzset{bipoles/vsourcesin/height=.5}
\ctikzset{bipoles/vsourcesin/width=.5}
\ctikzset{bipoles/cvsourceam/height=.6}
\ctikzset{bipoles/cvsourceam/width=.6}
%\ctikzset{tripoles/european controlled voltage source/width=.4}

\tikzstyle{every node}=[font=\footnotesize]
\tikzstyle{every path}=[line width=0.25pt,line cap=round,line join=round]
%\tikzstyle{every path}=[line cap=round,line join=round]


% Definição de cores MATLAB
\definecolor{matlab_blue}{rgb}	{         0,    0.4470,    0.7410}
\definecolor{matlab_orange}{rgb}{    0.8500,    0.3250,    0.0980}
\definecolor{matlab_yellow}{rgb}{    0.9290,    0.6940,    0.1250}
\definecolor{matlab_violet}{rgb}{    0.4940,    0.1840,    0.5560}
\definecolor{matlab_green}{rgb}	{	 0.4660,    0.6740,    0.1880}
\definecolor{matlab_lblue}{rgb}	{    0.3010,    0.7450,    0.9330}
\definecolor{matlab_red}{rgb}	{    0.6350,    0.0780,    0.1840}

% Personalização das legendas
\usepackage[format = hang,
			labelsep = endash,
			singlelinecheck = false,
			skip = 6pt,
			listformat = simple]{caption}

% Personalização das unidades
\sisetup{output-decimal-marker = {,}}
\sisetup{exponent-product = \cdot, output-product = \cdot}
\sisetup{tight-spacing=true}
\sisetup{group-digits = false}

% Personalizações de tipo de colunas de tabelas
\newcolumntype{L}[1]{>{\raggedright\let\newline\\\arraybackslash\hspace{0pt}}m{#1}}
\newcolumntype{C}[1]{>{\centering\let\newline\\\arraybackslash\hspace{0pt}}m{#1}}
\newcolumntype{R}[1]{>{\raggedleft\let\newline\\\arraybackslash\hspace{0pt}}m{#1}}

% Personalizações de cores da UDESC
\definecolor{CapaAmareloUDESC}{RGB}{243,186,83}		% Especializacao
\definecolor{CapaVerdeUDESC}{RGB}{0,112,52}			% Mestrado
\definecolor{CapaVermelhoUDESC}{RGB}{171,35,21}		% Doutorado
\definecolor{CapaAzulUDESC}{RGB}{38,54,118} 		% Pós-Doutorado


% Espaçamento depois do título
\setlength{\afterchapskip}{0.7\baselineskip}
% O tamanho do parágrafo é dado por:
\setlength{\parindent}{1.25cm}
% Controle do espaçamento entre um parágrafo e outro:
\setlength{\parskip}{0.15cm}  % tente também \onelineskip
%\SingleSpacing % Espaçamento simples
\OnehalfSpacing % Espaçamento 1,5 (UDESC/CCT)
%\DoubleSpacing	% Espaçamento duplo

% ---
% Margens - NBR 14724/2011 - 5.1 Formato
% ---
\setlrmarginsandblock{3cm}{2cm}{*}
\setulmarginsandblock{3cm}{2cm}{*}
\checkandfixthelayout[fixed]
% ---









\label{___:dados-do-trabalho}
% -----------------------------------------------------------------
% Informações de dados para CAPA e FOLHA DE ROSTO
% -----------------------------------------------------------------
\tipotrabalho{
    Trabalho de Conclusão de Curso
}

\titulo{
    Paralelos entre a realidade local e os programas de apoio a gestão escolar disponíveis pelo MEC
}%

% ATENÇÃO: O símbolo {} indica o sobrenome para a ficha catalográfica.
%          Exemplo: Sherlock Holmes {}da Silva para sobrenomes compostos;
%          Exemplo: Arnold Alois {}Schwarzenegger para sobrenome simples.
\autor{Francisco Lima {}Figueiredo}%
\newcommand\matricula{Matricula 202004137859}
\orientador{Maria Alejandra Iturrieta {}Leal}%
%\coorientador{}%
\instituicao{Universidade Estácio de Sá}%

\preambulo{
	Trabalho apresentada à professora Maria Alejandra Iturrieta Leal como requisito parcial para obtenção da aprovação na disciplina Educação Especial (CEL1694-3734363) 9004.
}

\local{Brasília}%
\data{\MONTH/\the\year}%
% ---










% compila o indice
\makeindex

% -----------------------------------------------------------------
% Início do documento
% -----------------------------------------------------------------
\begin{document}
	
	\selectlanguage{brazil}
	\frenchspacing            % Retira espaço extra obsoleto entre as frases.
	
	% Você pode comentar os elementos que não deseja em seu trabalho;

	% -----------------------------------------------------------------
	% ELEMENTOS PRÉ-TEXTUAIS
	% -----------------------------------------------------------------
	\pretextual

	\label{___:capa}
	% --- 
	% Capa
	% --- 
	%\imprimircapaTrabalho      % Capa UDESC para Trabalho
	\imprimircapaTCC			% Capa UDESC para TCC
	%\imprimircapaDissertacao	% Capa UDESC para Dissertações
	%\imprimircapaTese			% Capa UDESC para Teses
	%\imprimircapaPosDoc		% Capa UDESC para Pós-Doutorado

	%\imprimirCapaCustom{CapaAmareloUDESC}{\imprimirtipotrabalho}{\imprimirinstituicao}  %Capa UDESC para Pós-Doutorado

	%\imprimircapa				% Capa padrão


	% -----------------------------------------------------------------
	% ELEMENTOS PRÉ-TEXTUAIS
	% -----------------------------------------------------------------

		% ---
		% Folha de rosto
		% (o * indica que haverá a ficha bibliográfica)
		% ---
		\ImprimirSimOuNao[Não]{             % Elemento Obrigatório
			\imprimirfolhaderosto*                                                 
		}
		% ---




		\label{___:ficha-bibliográfica}
		% ---
		% Inserir a ficha bibliografica
		% ---
		\ImprimirSimOuNao[Sim]{             % Elemento Obrigatório

			% Isto é um exemplo de Ficha Catalográfica, ou ``Dados internacionais de
			% catalogação-na-publicação''. Você pode utilizar este modelo como referência.
			% Porém, provavelmente a biblioteca da sua universidade lhe fornecerá um PDF
			% com a ficha catalográfica definitiva após a defesa do trabalho. Quando estiver
			% com o documento, salve-o como PDF no diretório do seu projeto e substitua todo
			% o conteúdo de implementação deste arquivo pelo comando abaixo:

			% \begin{fichacatalografica}
			%     \includepdf{fig_ficha_catalografica.pdf}
			% \end{fichacatalografica}

			%	\setlength{\parindent}{0cm}
			%	\setlength{\parskip}{0pt}
			\begin{fichacatalografica}                            
				%\sffamily
				%\rmfamily
				\ttfamily \hbadness=10000
				\vspace*{\fill}					% Posição vertical
				\begin{center}					% Minipage Centralizado
			%	\begin{minipage}[c]{8cm}
			%	\centering \sffamily
			%	 Ficha catalográfica elaborada pelo(a) autor(a), com auxílio do programa de geração automática da Biblioteca Setorial do CCT/UDESC
			%	\end{minipage}
				\fbox{\begin{minipage}[c]{12.5cm}		% Largura
				\flushright
				{\begin{minipage}[c]{10.5cm}		% Largura
				\vspace{1.25cm}
				%\footnotesize
				\setlength{\parindent}{1.5em}
				\noindent \invertname{\imprimirautor} \par
				\imprimirtitulo{ }/{ }\imprimirautor. -- \imprimirlocal, \imprimirdata .\par
				\pageref{LastPage} p. : il. ; 30 cm.\par
				\vspace{1.5em}
				\imprimirorientadorRotulo~\imprimirorientador\par
				\imprimircoorientadorRotulo~\imprimircoorientador\par
				\imprimirtipotrabalho~--~\imprimirinstituicao, \imprimirlocal, \imprimirdata.\par
				\vspace{1.5em}
					1. Sociologia.
					2. Meio Ambiente.
					I. \invertname{\imprimirorientador}.
					II. \invertname{\imprimircoorientador}.
					III. \imprimirinstituicao. %
				\vspace{1.25cm}	%
				\end{minipage}%
				}%
				\end{minipage}}%
				\end{center}
			\end{fichacatalografica}
		
		\bigbreak
		
		}




		% ---
		% Inserir errata
		% ---
		\ImprimirSimOuNao[Não]{
			\begin{errata}
				Elemento opcional da \citeonline[4.2.1.2]{NBR14724:2011}. Exemplo:

				\vspace{\onelineskip}

				FERRIGNO, C. R. A. \textbf{Tratamento de neoplasias ósseas apendiculares com
				reimplantação de enxerto ósseo autólogo autoclavado associado ao plasma
				rico em plaquetas}: estudo crítico na cirurgia de preservação de membro em
				cães. 2011. 128 f. Tese (Livre-Docência) - Faculdade de Medicina Veterinária e
				Zootecnia, Universidade de São Paulo, São Paulo, 2011.

				\begin{table}[htb]
				\center
				\footnotesize
				\begin{tabular}{|p{1.4cm}|p{1cm}|p{3cm}|p{3cm}|}
				  \hline
				   \textbf{Folha} & \textbf{Linha}  & \textbf{Onde se lê}  & \textbf{Leia-se}  \\
					\hline
					1 & 10 & auto-conclavo & autoconclavo\\
				   \hline
				\end{tabular}
				\end{table}
				
			\end{errata}
			% ---
		}




		
		% ---
		% Inserir folha de aprovação
		% ---
		\ImprimirSimOuNao[Não]{
			\begin{folhadeaprovacao}
				\begin{center}
					{\MakeTextUppercase{\ABNTEXchapterfont\large\imprimirautor}}

					\vspace*{\fill} %\vspace*{\fill}
					\begin{center}
					  {\MakeTextUppercase{\ABNTEXchapterfont\bfseries\large\imprimirtitulo}}
					\end{center}
					\vspace*{\fill}
					
					%\hspace{.45\textwidth}
					{\begin{minipage}[c]{1\linewidth}
						\setlength{\parindent}{1.25cm}
						\imprimirpreambulo
					\end{minipage}}%
					\vspace*{\fill}
					\end{center}
						
					 
					{\bfseries Banca Examinadora: }
					\vspace*{\fill}
					
					{Orientador: \vspace{-16pt} }
					\assinatura{\textbf{Prof. \imprimirorientador , Dr.} \\ Univ. XXX} 
					{Coorientador: \vspace{-16pt}}   
					\assinatura{\textbf{Prof. \imprimircoorientador , Dr.} \\ Univ. XXX}
				   
					{Membros: \vspace{-16pt} } 
					
				% --- Exemplo de assinaturas em sequência ---       
					\setlength{\ABNTEXsignwidth}{8.5cm}
					
					\assinatura{\textbf{Prof. Professor, Dr.} \\ Univ. XXX}
					\assinatura{\textbf{Prof. Professor, Dr.} \\ Univ. XXX}
					\assinatura{\textbf{Prof. Professor, Dr.} \\ Univ. XXX}

				% --- Exemplo de assinaturas lado a lado ---
					\setlength{\ABNTEXsignwidth}{7.5cm}
				%
				%    \noindent\hfill\assinatura*{\textbf{Prof. Professor, Dr.} \\ Univ. XXX}%
				%    \hfill%
				%    \assinatura*{\textbf{Prof. Professor, Dr.} \\ Univ. XXX}%
				%    \hfill
				%    
				%    \noindent\hfill\assinatura*{\textbf{Prof. Professor, Dr.} \\ Univ. XXX}%
				%    \hfill%
				%    \assinatura*{\textbf{Prof. Professor, Dr.} \\ Univ. XXX}%
				%    \hfill
					
					\vspace*{\fill}  
					\begin{center}
					{\large\imprimirlocal, 01 de maio de \imprimirdata}
				\end{center}
				\vspace*{1cm}  
			\end{folhadeaprovacao}
		}
		% ---




		\label{___:dedicatória}
		% ---
		% Dedicatória
		% ---
		\ImprimirSimOuNao[Não]{
			\begin{dedicatoria}
			   \vspace*{\fill}
			   \centering
			   \noindent
			   \textit{ Este trabalho é dedicado às minhas maiores inspirações,\\
			   Minha esposa Daiane, minha filha Mayara Barros e minha sobrinha do coração Rachael Costa.} \vspace*{\fill}
			\end{dedicatoria}		
		}
		% ---
		
		
		
		
		\label{___:agradecimentos}
		% ---
		% Agradecimentos
		% ---
		\ImprimirSimOuNao[Não]{
			\begin{agradecimentos}
				Os agradecimentos principais são direcionados à Gerald Weber, Miguel Frasson, Leslie H. Watter, Bruno Parente Lima, Flávio de Vasconcellos Corrêa, Otavio Real Salvador, Renato Machnievscz\footnote{Os nomes dos integrantes do primeiro projeto abn\TeX\ foram extraídos de \url{http://codigolivre.org.br/projects/abntex/}} e todos aqueles que contribuíram para que a produção de trabalhos acadêmicos conforme as normas ABNT com \LaTeX\ fosse possível.

				Agradecimentos especiais são direcionados ao Centro de Pesquisa em Arquitetura da Informação\footnote{\url{http://www.cpai.unb.br/}} da Universidade de Brasília (CPAI), ao grupo de usuários \emph{latex-br}\footnote{\url{http://groups.google.com/group/latex-br}} e aos novos voluntários do grupo \emph{\abnTeX}\footnote{\url{http://groups.google.com/group/abntex2} e \url{http://www.abntex.net.br/}}~que contribuíram e que ainda contribuirão para a evolução do \abnTeX.
			\end{agradecimentos}
		}
		% ---




		\label{___:epigrafe}
		% ---
		% Epígrafe
		% ---
		\ImprimirSimOuNao[Não]{
			\begin{epigrafe}
				\vspace*{\fill}
				\hspace{.35\textwidth}
				{\begin{minipage}{.6\textwidth}
					\begin{flushright}
						\textit{``Porque eu fazia do amor um cálculo matemático errado: pensava que, somando as compreensões, eu amava. Não sabia que, somando as incompreensões é que se ama verdadeiramente.``\\
							(Clarice Lispector)}
					\end{flushright}
				\end{minipage}}%
			\end{epigrafe}
		}
		% ---
		
		
		
		
		\label{___:resumo}
		% ---
		% RESUMOS
		% ---
		\ImprimirSimOuNao[Não]{
			% resumo em português
			\setlength{\absparsep}{18pt} % ajusta o espaçamento dos parágrafos do resumo
			\begin{resumo}
			 O presente estudo tem como objetivo coletar diversas estatísticas, parâmetros e métricas de provas do ENEM de vários anos e discorrer sobre a prova, o desempenho na disciplina de matemática e os eventuais fatores e eventos correlatos que possam indicar causa-efeito que afetem a nota final da disciplina na avaliação nacional. \\
			 Foram analisadas provas entre os anos de 2015 a 2019 do ENEM, seus gabaritos e os microdados de cada aluno, o que fornece um ambiente rico para perguntas e testes de hipóteses.\\
			 

			 \textbf{Palavras-chave}: ENEM, estatísticas, habilidades, competências, métricas, parâmetros.
			\end{resumo}
		}
		
		
		
		
		\label{___:abstract}
		% ---
		% Abstract
		% ---
		\ImprimirSimOuNao[Não]{
			% resumo em inglês
			\begin{resumo}[Abstract]
			 \begin{otherlanguage*}{english}
			   This is the english abstract.

			   \vspace{\onelineskip}
			 
			   \noindent 
			   \textbf{Keywords}: latex. abntex. text editoration.
			 \end{otherlanguage*}
			\end{resumo}
		}
		



		\label{___:listas}
		% ---
		% inserir lista de ilustrações
		% ---
		\ImprimirSimOuNao[Não]{
			\pdfbookmark[0]{\listfigurename}{lof}
			\listoffigures*
			\cleardoublepage
		}
		% ---





		% ---
		% inserir lista de tabelas
		% ---
		\ImprimirSimOuNao[Não]{
			%\pdfbookmark[0]{\listtablename}{lot}
			%\listoftables*
			%\cleardoublepage
		}
		% ---





		% ---
		% inserir lista de abreviaturas e siglas
		% ---
		\ImprimirSimOuNao[Não]{
			%\begin{siglas}
			%  \item[ABNT] Associação Brasileira de Normas Técnicas
			%  \item[abnTeX] ABsurdas Normas para TeX
			%\end{siglas}
		}
		% ---





		% ---
		% inserir lista de símbolos
		% ---
		\ImprimirSimOuNao[Não]{
			%\begin{simbolos}
			%  \item[$ \Gamma $] Letra grega Gama
			%  \item[$ \Lambda $] Lambda
			%  \item[$ \zeta $] Letra grega minúscula zeta
			%  \item[$ \in $] Pertence
			%\end{simbolos}
		}
		% ---



		\label{___:sumario-índice}
		% ---------------------
		% inserir o sumario    
		% ---------------------
		\ImprimirSimOuNao[Sim]{
			\newpage
			\pdfbookmark[0]{\contentsname}{toc}

			% Formatação forçada do sumário identado e dot juntinho
			\makeatletter 
				\renewcommand{\cftdotsep}{.5}
				\renewcommand{\cftlastnumwidth}{1em}
				\cftsetindents{chapter}{1em}{1.5em}
				\cftsetindents{section}{1.5em}{2em}
				\cftsetindents{subsection}{3.75em}{2.5em}
				\cftsetindents{subsubsection}{6.2em}{3.5em}
			\makeatother

			\tableofcontents*

			\cleardoublepage
		}
		
		




















		
% -----------------------------------------------------------------
% ELEMENTOS TEXTUAIS
% -----------------------------------------------------------------
\textual
		
\chapter{Objetivos}

	O objetivo do presente trabalho é realizar reflexões e traçar paralelos entre a disponibilização de programas de apoio a gestão e a realidade local do aluno.



\chapter{Introdução}

	Como alunos matriculados em cadeiras de formação de professores, entendo que ter uma visão 360º do todo da gestão escolar é fundamental para a plena formação do professor e para a criação de empatia mínima com todo o corpo escolar, notadamente para quem trabalha nos "bastidores".
	
	
			
\chapter{Procedimentos Metodológicos}

	\section{Pesquisa Bibliográfica} 

		Após pesquisa no site do MEC em \url{http://portal.mec.gov.br/pradime} selecionamos o programa PRADIME.

		Este programa atua no âmbito municipal, com foco nos dirigentes, na equipe técnica da rede municipal de ensino e nas diversas dimensões da gestão educacional, buscando propiciar um espaço permanente de formação, troca de experiências e acesso a informações que colaborem com a melhora na qualidade da educação básica.
		
		O principal objetivo deste programa é contribuir para o avanço da educação em relação às metas do Plano Nacional de Educação (PNE) e do Plano de Desenvolvimento da Educação (PDE).

		Como cita o próprio site:
		
		\begin{citacao}
			O Programa de Apoio aos Dirigentes Municipais de Educação (Pradime), parceria do Ministério da Educação com a União Nacional dos Dirigentes Municipais (UNDIME), foi criado com o objetivo de fortalecer e apoiar os dirigentes da educação municipal na gestão dos sistemas de ensino e das políticas educacionais. O intuito do programa é contribuir para o avanço em relação às metas e aos compromissos do Plano Nacional de Educação (PNE) e do Plano de Desenvolvimento da Educação (PDE).
			
			O objetivo é oferecer a todos os dirigentes municipais de educação e as equipes técnicas que atuam na gestão da educação e do sistema municipal, um espaço permanente de formação, troca de experiências, acesso a informações sistematizadas e à legislação pertinente, que ajude a promover a qualidade da educação básica nos sistemas públicos municipais de ensino, focando as diversas dimensões da gestão educacional.
			
			O PRADIME desenvolve dois tipos principais de atividade: encontros presenciais e curso a distância. A primeira propicia a participação dos dirigentes municipais em encontros com representantes do MEC, do MEC/FNDE e da UNDIME, dentre outros, onde são discutidos diversos programas e temas relacionados à política educacional. Neles são realizadas palestras, oficinas e também apresentações de exemplos bem sucedidos de gestão da educação municipal.
			
			A segunda iniciativa, o curso a distância, é um espaço de aperfeiçoamento e formação dos dirigentes municipais de educação em nível de extensão e, em alguns casos, especialização. O curso aborda as diversas temáticas que estão sob sua responsabilidade, abrangendo o planejamento e a avaliação do sistema educacional, o financiamento e a gestão orçamentária, a infraestrutura física e a logística de suprimentos bem como a gestão de pessoas, considerando o ambiente de governança democrática. Neste espaço virtual de aprendizagem, além do curso propriamente dito, o aluno ainda encontrará um espaço propício para o intercâmbio de idéias e experiências, contando com o apoio e orientação de professores consultores.
		\end{citacao}

		Para a pesquisa, foram lidos os 3 tomos dos cadernos de texto distribuídos pelo PRADIME \nocite{PRADIME01} \nocite{PRADIME02} \nocite{PRADIME03}

\chapter{Resultados}

	\section{Reflexões}

		É demais interessante para um país ter um programa de formação de gestores, até porque nenhum político ou governo jamais deixaria a educação como prioridade.
		
		Um trecho de \citeaa{PRADIME01} é muito esclarecedor:
		
		\begin{citacao}
			Porém, isso não pode minimizar a importância do papel estratégico da ação educacional em nível municipal, que tem como função precípua o cumprimento efetivo do direito à educação (que não se reduz à garantia do acesso à escola). Para isso, é preciso que a capacidade e a efetividade dos processos decisórios, no que se refere ao dirigente municipal, sejam desenvolvidas e fortalecidas.
		\end{citacao}
		
		No entanto não é isso que acontece, basta citar um trecho de \citeaa{Glap2017}
		
		\begin{citacao}
			Ressalte-se, também, os problemas decorrentes das eleições municipais de
			2012, com a mudança de prefeitos em alguns municípios parceiros. Verificou-se
			descontinuidade administrativa, implicando em dificuldades de articulação com os
			novos gestores e suas equipes técnicas, muitos dos quais revelaram completo
			desconhecimento sobre o funcionamento do PARFOR, como inserção de demanda,
			pré-inscrição, validação dos candidatos, entre outros. Em alguns casos, não se tinha
			conhecimento até mesmo da senha de acesso à Plataforma Freire.
		\end{citacao}
	
		Enquanto a educação não for programa de estado de fato e de direito, ele está a mercê de pessoas que usam a eleição como cabide eleitoral.
		
		Até onde sei, o programa está desvinculada de metas objetivas atreladas a melhoria da gestão escolar após a capacitação dos gestores. 
		

	\section{Conclusão}		
		Concluímos  que  a  Gestão  Escolar  envolve  vários  setores,  dentre  eles  tem  a  gestão pedagógica,  administrativa  e  financeira. Todas  precisam  trabalhar  em  conjunto,  para o  bom  funcionamento  da  escola.  O  gestor  interage  com  todos  os setores  em  prol  do desempenho escolar.
		
		Entretanto, ainda carecemos de mecanismos que permitam que educação seja tratado como dever de estado. Notadamente programas e recursos mudam de governo a governo.
		

	\section{Evoluções e melhoramentos desejados}
	
		Seria interessante uma ampla pesquisa, comandado pelo INEP, no sentido de identificar os impactos do PRADIME nas gestões municipais em que o programa foi aplicado, em comparação onde não foi realizado.
		Cabe ressaltar que essa pesquisa deve, acima de tudo, clusterizar os municípios por porte e quantidade de habitantes, para poder haver paralelo de comparação onde o programa foi aplicado ou não.





\label{___:bibliografia}
%%---------------------------------------------------------------------
%% BIBLIOGRAFIA - Referências Bibliográficas
%%---------------------------------------------------------------------
% Referências bibliográficas
\bibliography{bibliografia/!!!bibliografia}	        % Elemento Obrigatório








\postextual

	% ----------------------------------------------------------
	% Glossário
	% ----------------------------------------------------------

	% Consulte o manual da classe abntex2 para orientações sobre o glossário.

	\ImprimirSimOuNao[Não]{             % Elemento Opcional
		\glossary
	}



	% ----------------------------------------------------------
	% Apêndices
	% ----------------------------------------------------------

	\ImprimirSimOuNao[Nao]{             % Elemento Opcional
		% ---
		% Inicia os apêndices
		% ---
		\begin{apendicesenv}

		% Imprime uma página indicando o início dos apêndices
		\partapendices

		% ----------------------------------------------------------
		\chapter{Sobre esse trabalho}
		% ----------------------------------------------------------


			Esse documento foi produzido e programado usando-se \LaTeX, MikTeX, abntex2 e todo conteúdo possui links referencias clicáveis, sejam tabelas, figuras, imagens de vídeos, autores com seu respectivo registro bibliográfico.
			Informamos que o código fonte do projeto gerador desse PDF está disponível no endereço abaixo (legível também pelo QR Code abaixo): \\
			
			\qrset{link, height=4cm}
			\begin{center}
				\href{https://github.com/ChicoFigueiredo/Estacio-TCC-Estacio-Matematica}{
					\qrcode{https://github.com/ChicoFigueiredo/Estacio-TCC-Estacio-Matematica}
				}
			\end{center}
			\begin{center}
				{\tiny \url{https://github.com/ChicoFigueiredo/Estacio-TCC-Estacio-Matematica} }
			\end{center}
			


		% ----------------------------------------------------------
		%\chapter{Nullam elementum urna vel imperdiet sodales elit ipsum pharetra ligula
		%ac pretium ante justo a nulla curabitur tristique arcu eu metus}
		%% ----------------------------------------------------------
		%\lipsum[55-57]

		\end{apendicesenv}
		% ---
	}

	\label{___:anexos}
	% ----------------------------------------------------------
	% Anexos
	% ----------------------------------------------------------
	%
	% ---
	% Inicia os anexos
	% ---
	\ImprimirSimOuNao[Não]{             % Elemento Opcional
		\begin{anexosenv}

			% Imprime uma página indicando o início dos anexos
			\partanexos
			
			\makeatletter 
				\def\thesection{\alph{section}}
				\setcounter{chapter}{1}
			\makeatother
			
			\section{Conteúdo desse trabalho}


				Esse documento foi programado em \LaTeX, MikTeX, abntex2 e todo conteúdo possui links referenciais clicáveis, sejam tabelas, figuras, imagens de vídeos, autores com seu respectivo registro bibliográfico.
				Informamos que projeto gerador desse PDF está disponível no endereço (legível também pelo QR Code abaixo): \\
				\url{https://github.com/ChicoFigueiredo/estacio-Trab001-AASE-202004137859.git} \\
				\qrset{link, height=4cm}
				\begin{center}
					\href{https://github.com/ChicoFigueiredo/estacio-Trab001-AASE-202004137859.git}{
						\qrcode{https://github.com/ChicoFigueiredo/estacio-Trab001-AASE-202004137859.git}
					}
				\end{center}


				Apresentação no OneDrive: \url{https://1drv.ms/p/s!AgRBucATAhUblzAldnG4LGnWNV-r?e=ykIvGi} \\
				\begin{center}
					\href{https://1drv.ms/p/s!AgRBucATAhUblzAldnG4LGnWNV-r?e=ykIvGi}{
						\qrcode{https://1drv.ms/p/s!AgRBucATAhUblzAldnG4LGnWNV-r?e=ykIvGi}
					}
				\end{center}

				Vídeo no YouTube: \url{https://youtu.be/szsZ_Uuk1zk} \\
				\begin{center}
					\href{https://youtu.be/szsZ_Uuk1zk}{
						\qrcode{https://youtu.be/szsZ_Uuk1zk}
					}
				\end{center}


				\section{Códigos e Programas utilizados}

				\subsection{Programas em Python}
					programas em python

				\subsection{Programas em R}
					programas em r

				\section{Lista de questões}
					lista de questões

			
		\end{anexosenv}
	}
	
	
	%%---------------------------------------------------------------------
	%% INDICE REMISSIVO
	%%---------------------------------------------------------------------
	\ImprimirSimOuNao[Não]{             % Elemento Opcional
		\phantompart
		\printindex
	}
	%---------------------------------------------------------------------


\end{document}

% -----------------------------------------------------------------
% Fim do Documento
% -----------------------------------------------------------------
